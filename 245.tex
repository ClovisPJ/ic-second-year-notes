\documentclass[twocolumn,english]{article}
\usepackage[latin9]{inputenc}
\usepackage[landscape]{geometry}
\geometry{verbose,tmargin=0.5in,bmargin=0.75in,lmargin=0.5in,rmargin=0.5in}
\setlength{\parskip}{0bp}
\setlength{\parindent}{0pt}
\usepackage{amssymb}

\makeatletter



\usepackage{array}
\usepackage{multirow}
\usepackage{amsbsy}




\providecommand{\tabularnewline}{\\}

\setlength{\columnsep}{0.25in}
\usepackage{xcolor}
\usepackage{textcomp}
\usepackage{listings}
\lstset{
  tabsize=2,
  basicstyle=\small\ttfamily,
}



\usepackage{babel}
\usepackage{listings}
\renewcommand{\lstlistingname}{Listing}

\makeatother

\usepackage{babel}
\begin{document}

\title{Reference Sheet for C245 Probability and Statistics}

\date{Autumn 2017}
\maketitle

\section{Probability}
\begin{itemize}
\item \emph{Sample space} $S$: Range of possible outcomes of a random experiment.
\item \emph{Event}: Subset of sample space.
\begin{itemize}
\item \emph{Null event}: $\emptyset$.
\end{itemize}
\item Events are \emph{mutually exclusive} if $\forall i,j.E_{i}\cap E_{j}=\emptyset$.
\end{itemize}

\paragraph{The $\sigma$-algebra}

A collection $\mathfrak{S}$ of subsets of $S$ is a $\sigma$-field
or $\sigma$-algebra if it satisfies:
\begin{enumerate}
\item $\emptyset\in\mathfrak{S}$.
\item Closed under countable union: if $E_{1},E_{2},\dots\in\mathfrak{S}$
then $\cup_{i=1}^{\infty}E_{i}\in\mathfrak{S}$.
\item Closed under complements: if $E\in\mathfrak{S}$ then $\overline{E}\in\mathfrak{S}$.
\end{enumerate}
\emph{Properties}:
\begin{enumerate}
\item $P\left(\emptyset\right)=0$.
\item $P\left(\overline{E}\right)=1-P\left(E\right)$.
\item $P\left(E\cup F\right)=P\left(E\right)+P\left(F\right)-P\left(E\cap F\right)$.
\end{enumerate}

\end{document}
